\documentclass[conference]{IEEEtran}
\IEEEoverridecommandlockouts
% The preceding line is only needed to identify funding in the first footnote. If that is unneeded, please comment it out.
\usepackage{cite}
\usepackage{amsmath,amssymb,amsfonts}
\usepackage{algorithmic}
\usepackage{graphicx}
\usepackage{textcomp}
\usepackage{xcolor}
\def\BibTeX{{\rm B\kern-.05em{\sc i\kern-.025em b}\kern-.08em
    T\kern-.1667em\lower.7ex\hbox{E}\kern-.125emX}}
\begin{document}

\title{Identification of Digits from Sign Language Images}

\author{\IEEEauthorblockN{José Santos, 98279}
\IEEEauthorblockA{\textit{DETI} \\
\textit{Universidade de Aveiro}}
\and
\IEEEauthorblockN{Henrique Sousa, 98324}
\IEEEauthorblockA{\textit{DETI} \\
\textit{Universidade de Aveiro}}
}

\maketitle

\begin{abstract}
    The purpose of this work is to implement and
    compare machine learning models capable of identifying digits from sign language images.
    In this paper, we tried to obtain a good result with the models using
    a dataset provided by Kaggle. Some changes are discussed,
    based on the work of others that positively affected our
    work.
\end{abstract}

\begin{IEEEkeywords}
Sign language recognition, Digit recognition, Machine learning
\end{IEEEkeywords}

\section{Introduction}
In recent years, there has been growing interest in using machine learning to develop computer vision systems capable of recognizing sign language gestures. Such systems could be used to improve communication between hearing and non-hearing individuals, as well as to facilitate the development of new technologies for the deaf and hard-of-hearing community.

In this paper, we present a novel approach to digit recognition from sign language images using machine learning. We explore several different models, including neural networks, support vector machines, and decision trees, and compare their performance on a dataset of sign language images. We also investigate the impact of some preprocessing techniques.

\section{State of the art}

\section{Dataset Preprocessing}
\section{Dataset Analysis}

\section{Models}

\section{Conclusion}

\section{References}

\begin{thebibliography}{00}
    \bibitem{b1} Akanksha Telagamsetty, Sign Language Digits Classification https://medium.com/analytics-vidhya/sign-language-classification-64fe8ad0fc2c
\end{thebibliography}

\end{document}
